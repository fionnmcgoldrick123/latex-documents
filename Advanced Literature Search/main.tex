\documentclass[conference]{IEEEtran}
\usepackage{graphicx} % Required for inserting images
\usepackage{hyperref}
\usepackage{biblatex}
\usepackage{listings}
\usepackage{pbalance}


\addbibresource{refs.bib}

\title{Advanced Literature Search}
\author{Fionn McGoldrick}
\date{November 2025}

\begin{document}

\maketitle

\section{Introduction}

This document is a sample solution for Tutorial Exercise 5.
Its brief is available as a PDF on our Moodle page.

Conducting effective literature searches is a fundamental
research skill that requires mastery of advanced query techniques. This exercise demonstrates how to use sophisticated
search queries in three major academic databases: IEEE
Xplore, ACM Digital Library, and Scopus. These techniques
include Boolean operators, proximity searches, wildcard characters, field-specific filters, and strategic exclusions to refine
results. The exercise also illustrates strategies for finding
both secondary (reviews and surveys) and primary sources,
depending on your research objectives.

Each section presents a specific research topic and shows
how to progressively refine search queries through iterative
improvements, demonstrating how researchers narrow from
hundreds or thousands of results to a manageable set of highly
relevant papers. The exercises use real query syntax that you
can copy and execute directly in each database.

\section{IEEE XPLORE}

This research focuses on security architectures and frameworks for Internet of Things deployments in smart city
applications, specifically targeting survey and review papers
while excluding healthcare and honeypot-related cybersecurity
research. The topic “IoT Security Architectures for Smart
Cities” narrows the broad “Internet of Things” domain to
security concerns in urban infrastructure contexts.

\subsection{Initial Broad Search}

This initial query searches for papers with either “Internet
of Things” or its abbreviation “IoT” in the document title,
combined with “smart cit*” (where the wildcard * captures
“city” and “cities”):

\begin{lstlisting}
(("Document Title":"Internet of Things")
OR ("Document Title":"IoT"))
AND ("Document Title":"smart cit*")
\end{lstlisting}

The use of field-specific searches (e.g. \texttt{"Document
Title"}:) ensures that these core concepts appear in the title
rather than just anywhere in the paper, which significantly improves relevance. However, this search returns approximately
974 results covering all aspects of IoT in smart cities, including
applications, protocols, and infrastructure, making it too broad
for a focused security review.

\subsection{Adding Review Focus}

The second query adds a critical refinement by restricting
results to survey and review papers:
\begin{lstlisting}
(("Document Title":"Internet of Things")
OR ("Document Title":"IoT"))
AND ("Document Title":"smart cit*")
AND (("Document Title":"Literature Review")
OR ("Document Title":"survey")
OR ("Document Title":"review"))
\end{lstlisting}

By requiring “Literature Review”, “survey”, or “review”
to appear in the document title, we filter for secondary
sources that synthesise existing research rather than primary
research papers reporting original studies. This is particularly
valuable for literature review work, as these papers provide
comprehensive overviews of a research area and often identify
key themes, gaps, and future directions. This single refinement
dramatically reduces the results from 974 to approximately
44 papers, demonstrating the power of field-specific filters.
The results now focus on papers that specifically aim to
review the IoT smart city landscape, but still cover all aspects
(applications, protocols, security, etc.) without a specific focus
area.

\subsection{Final Refined Search with Security Focus and Exclusions}

The final query adds two sophisticated refinement techniques that narrow the focus to security architectures while excluding unwanted subdomains:
\begin{lstlisting}
(("Document Title":"Internet of Things")
OR ("Document Title":"IoT"))
AND ("Document Title":"smart cit*")
AND (("Document Title":"Literature Review")
OR ("Document Title":"survey")
OR ("Document Title":"review"))
AND ("security" NEAR/10 "architecture")
AND (NOT ("Document Title":"Honeypot*"))
AND (NOT ("Document Title":"Health*"))
\end{lstlisting}

First, the proximity operator (NEAR/10) requires that “security” and “architecture” appear within 10 words of each other
anywhere in the document. This ensures the paper addresses
security architectural concerns rather than mentioning these
terms in unrelated contexts or separate sections. The proximity
constraint is crucial because papers might mention security in
one paragraph and architecture in another, unrelated paragraph,
which would not indicate a focus on security architectures.

Second, strategic exclusions remove unwanted research directions using NOT operators with wildcards. Papers with
“Honeypot*” in the title (where * captures variations like
“Honeypots” or “Honeypotting”) are excluded because honeypot research focuses on decoy systems for attracting attackers rather than production security architectures for smart
cities. Similarly, “Health*” is excluded to remove healthcare
IoT papers (covering “Health”, “Healthcare”, “Healthtech”),
which have fundamentally different security requirements and

\begin{figure}[ht]
    \centering
    \includegraphics[width=\columnwidth]{ss1.png}
    \caption{IEEE Xplore search results for the refined query on IoT security
    architectures for smart cities. The results show survey and review papers
    addressing security architectural frameworks while excluding healthcare and
    honeypot research.
    }
    \label{fig:IEEEXplore}
\end{figure}

regulatory contexts (e.g., HIPAA compliance) than those of
smart city infrastructure. These combined refinements reduce
results from 44 to approximately 3 highly targeted papers,
demonstrating how proximity operators and exclusions can
achieve laser-focused results for specific research questions.
Figure \ref{fig:IEEEXplore} shows the search results from the most refined
query. The retrieved papers specifically address security architecture concerns in IoT smart city deployments through survey
and review methodologies, such as \cite{10496658}, demonstrating how
systematic use of field filters, proximity operators, wildcards,
and exclusions produces highly targeted results.

\section{ACM DIGITAL LIBRARY:}

This research investigates the application of machine learning and artificial intelligence techniques to automate various
aspects of software testing, including test case generation,
test automation, and defect prediction. The topic “Machine
Learning Techniques for Automated Software Testing” is
highly relevant to modern software development practices,
where AI-driven approaches increasingly augment traditional
testing methodologies to improve efficiency and test coverage.

\subsection{ Initial Broad Search}

This initial query casts a wide net by combining multiple
alternative terms for machine learning and AI techniques with
various testing-related terms:

\begin{lstlisting}
("machine learning" OR "deep learning" OR
"artificial intelligence" OR AI OR ML)
AND (testing OR "test automation" OR
"software testing")
\end{lstlisting}

The use of OR operators within parentheses creates synonym groups, ensuring we capture papers regardless of which
terminology authors prefer. Including both full terms (e.g.,
“machine learning”) and common abbreviations (e.g., ML,
AI) is important because different research communities use
different conventions. The query uses quotation marks around
multi-word phrases to ensure they are searched as units. This
broad search returns approximately 99,999 results, covering
all aspects of AI/ML in testing contexts, including papers where testing is mentioned only tangentially or where ML is
a minor component. This initial search is intentionally comprehensive to understand the full scope of available literature
before applying refinements.

\subsection{Adding Field Restrictions and Software Context}
The second query adds field-specific restrictions to improve
precision dramatically:
\begin{lstlisting}
Title:("machine learning" OR "deep learning"
OR AI OR ML) AND (test* OR testing OR
software)
\end{lstlisting}

By requiring ML/AI terms to appear in the Title: field,
we ensure these techniques are the primary focus of the paper
rather than just mentioned in passing. Papers with ML/AI
in the title are fundamentally about these approaches, not
papers that merely reference them in related work sections.
The wildcard test* captures variations including “test”, “tests”,
“testing”, and “tester”, providing flexibility without sacrificing
precision. Adding the general term “software” ensures we
maintain a software engineering context rather than ML/AI
applications in other domains, such as medical diagnosis
or financial prediction. This refinement reduces results from
99,999 to approximately 15,754 papers, a dramatic reduction
that demonstrates the power of field-specific searching. The
remaining papers genuinely focus on ML/AI as the main
contribution to software testing problems.

\subsection{Final Refined Search with Proximity and Exclusions}
The final query applies two additional powerful refinement
techniques that transform an already focused set of papers into
a highly targeted collection suitable for in-depth review:

\begin{lstlisting}
Title:("machine learning" OR "deep learning"
OR AI OR ML) AND Title:(test*) AND
("test generation"~10 OR
"test automation"~10 OR
"defect prediction"~10) AND NOT
(survey OR review OR blockchain OR IoT)
\end{lstlisting}

First, proximity operators (ACM’s tilde notation ˜10)
require that specific concept pairs appear within 10
words of each other, indicating these terms are discussed together as unified concepts rather than mentioned
separately. For example, "\texttt{test generation}"˜10 ensures papers specifically address the generation of test
cases using ML, not papers that separately mention testing in one section and generation in another unrelated context. Similarly, "\texttt{test automation}"˜10 and
"\texttt{defect prediction}"˜10 focus results on specific testing applications rather than general testing discussions. This
proximity requirement also adds a Title: restriction on
test*, ensuring test-related concepts appear prominently in the
paper’s title

\begin{figure}[ht]
    \centering
    \includegraphics[width=\columnwidth]{ss2.png}
    \caption{ACM Digital Library search results for the refined query on
            machine learning techniques for automated software testing. The results show
            primary research papers applying ML/AI specifically to test generation, test
            automation, or defect prediction, excluding surveys and specialised application
            domains.
    }
    \label{fig:ACM}
\end{figure}

that synthesises existing work (which would be relevant for
a different research question but not for understanding specific ML techniques for testing). Additionally, “blockchain”
and “IoT” are excluded because papers focused on these
specialised domains have different constraints, architectures,
and concerns than general software testing applications. These
combined refinements reduce results from 15,754 to approximately 8 highly targeted papers, a substantial reduction
that demonstrates how proximity searching, combined with
strategic exclusions, produces laser-focused results. The final
set comprises papers that specifically apply ML/AI techniques
to concrete software testing challenges with primary research
contributions.

As shown in Figure \ref{fig:ACM}, the refined search returns papers that
specifically address machine learning applications to concrete
software testing challenges, such as \cite{10186485}. The progression from
over 99,999 results to just 8 demonstrates how systematic application of field restrictions, proximity operators, and
strategic exclusions transforms an overwhelming literature
landscape into a manageable, highly relevant set of papers
suitable for in-depth review.

\section{Scopus}
This research investigates agile methodologies and practices
specifically adapted for geographically distributed software
development teams, focusing on primary research papers (empirical studies, case studies, experiments) that report original
findings about methodological practices. The topic “Agile
Software Development Methodologies for Distributed Teams”
addresses the challenges and solutions to agile practices
when teams cannot co-locate, an increasingly relevant concern
in modern software engineering, particularly in the postpandemic era, when remote work has become prevalent.

\subsection{Initial Broad Search}
This initial Scopus query uses the TITLE-ABS-KEY field
code to search within titles, abstracts, and author-specified
keywords simultaneously:

\begin{lstlisting}
TITLE-ABS-KEY(agile AND
"distributed team*" AND
"software development")
\end{lstlisting}

The query combines three core concepts: agile methodologies, distributed teams (with wildcard team* capturing
both “team” and “teams”), and software development context.
Scopus syntax differs significantly from IEEE Xplore and
ACM, requiring specific field codes and formatting conventions. Scopus requires explicit Boolean operators (AND) to
combine multiple search terms within field codes, and uses
quotation marks to search for exact phrases. This broad search
returns approximately 152 results covering various aspects of
distributed agile development, including case studies, surveys,
experience reports, theoretical papers, and tool demonstrations.
While this provides a comprehensive view of the literature
landscape, it is too diverse and includes too many secondary
sources for a focused literature review based primarily on
original research.

\subsection{Adding Field Restrictions}
The second query adds field-specific restrictions to improve
precision:

\begin{lstlisting}
TITLE(agile) AND
TITLE-ABS-KEY("distributed team*") AND
TITLE-ABS-KEY("software development")
\end{lstlisting}

By requiring “agile” to appear specifically in the TITLE
field rather than just anywhere in the title-abstract-keywords
combination, we ensure that agile methodologies are the
primary focus of the paper rather than a secondary topic or
mentioned only in passing. Papers with “agile” in the title are
fundamentally about agile practices, not papers that merely
reference agile briefly in their related work or background sections. The distributed teams and software development terms
remain in TITLE-ABS-KEY to maintain reasonable recall
while ensuring these concepts are prominent. This refinement
reduces the results from 152 to approximately 91 papers,
a significant reduction that demonstrates how field-specific
restrictions can effectively narrow results by ensuring that
core concepts appear in prominent document locations. The
remaining papers centre on agile as their main contribution,
though they still include both primary and secondary research
and may not all focus specifically on methodological practices.

\subsection{ Final Refined Search with Proximity, Temporal Filters, and
Exclusions}
The final query applies multiple sophisticated refinement
techniques that showcase Scopus’s advanced capabilities and
transform a moderately focused set into a highly targeted
collection of recent primary research:

\begin{lstlisting}
TITLE(agile) AND
TITLE-ABS-KEY("distributed" W/5 "team*") AND
TITLE-ABS-KEY("software" W/3 "development")
AND TITLE-ABS-KEY(method* OR practice*) AND
PUBYEAR > 2019 AND NOT TITLE(blockchain OR
review OR survey OR "systematic literature"
OR "literature review")
\end{lstlisting}

First, proximity operators (Scopus’ W/n notation) and wildcards work together to ensure genuine conceptual relationships. The W/5 operator means “distributed” and “team*” must
appear within 5 words of each other (in any order), indicating
papers discussing distributed teams as a unified concept rather
than mentioning these terms separately in different contexts.
Similarly, W/3 ensures “software” and “development” appear
within 3 words, maintaining the software engineering context.
The wildcards method* and practice* capture variations such
as “methodology”, “methodologies”, “methods”, “practice”,
“practices”, and “practitioner”, ensuring that papers address
methodological or practical aspects rather than just theoretical
discussions.

Second, temporal filters and strategic exclusions combine
to focus on recent primary research while removing unwanted
paper types and specialised domains. The temporal filter
(\texttt{PUBYEAR > 2019}) restricts results to publications from
2020 onwards, ensuring the review reflects current practices,
which is particularly important in rapidly evolving fields like
distributed agile development, where practices have changed
significantly, especially following the COVID-19 pandemic’s
impact on remote work. Strategic NOT exclusions in the
title remove multiple unwanted categories: “review”, “survey”,
“systematic literature”, and “literature review” are excluded
to focus on primary research papers (case studies, empirical
studies, experiments) rather than secondary literature, aligning
with the principle that literature reviews should be based
primarily on original research contributions. Additionally,
“blockchain” is excluded to remove papers where agile is
discussed only in the context of this specialised domain.
These combined refinements reduce the results from 91 to
approximately 29 highly targeted recent primary research papers, a noticeable reduction that demonstrates how proximity
operators, temporal filters, and strategic exclusions together
produce laser-focused results suitable for a comprehensive yet
manageable literature review.

Figure 3 presents the search results from Scopus for
the most refined query. The retrieved papers, such as \cite{Javed2025EnhancingRM},
specifically report original research on agile methodological
concerns in distributed team contexts, demonstrating how
database-specific syntax (W/n proximity, TITLE vs TITLEABS-KEY distinctions), field restrictions, wildcards, temporal
filters (PUBYEAR), and strategic exclusions (AND NOT)
produce highly focused results centered on primary research
suitable for evidence-based literature reviews.


\begin{figure}[ht]
    \centering
    \includegraphics[width=\columnwidth]{ss3.png}
    \caption{Scopus search results for the refined query on agile methodologies
    for distributed software development teams. The results show recent primary
    research papers addressing methodological practices for geographically dispersed agile teams, excluding secondary literature and specialised domains.
    }
    \label{fig:IEEEXplore}
\end{figure}

\section{Conclusion}

This exercise illustrated how to conduct effective literature
searches using advanced query techniques across three major
academic databases. The critical lessons are:
\begin{itemize}
    \item Use proximity operators to ensure related concepts appear near each other, indicating genuine relationships rather than coincidental mentions.
    \item Use wildcards to capture word variations and improve recall without sacrificing precision.
    \item Target specific fields to ensure core concepts appear in prominent locations, significantly improving relevance.
    \item Use exclusions strategically (NOT) to remove unwanted subdomains or contexts that would otherwise dilute results with irrelevant papers.
    \item Include synonyms and related terms to avoid missing relevant papers that use different vocabulary or terminology.
    \item Combine Boolean operators strategically using AND, OR, and NOT to express complex search requirements.
    \item Adapt queries to each database's unique syntax requirements and field codes.
    \item Apply publication year restrictions, as currency is important for rapidly evolving topics.
    \item Start broadly to understand the literature landscape, then systematically narrow through multiple query iterations, aiming for one to a few dozen highly relevant papers for a focused literature review.
\end{itemize}


\printbibliography

\end{document}
