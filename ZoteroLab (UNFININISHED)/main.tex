\documentclass[conference]{IEEEtran}
\usepackage{graphicx} % Required for inserting images
\usepackage[style=ieee]{biblatex}
\usepackage{hyperref}

\addbibresource{refs.bib}

\title{Research Methods - Tutorial 4}
\author{Fionn McGoldrick}
\date{October 2025}

\begin{document}

\maketitle

\begin{abstract}
Abstract—Federated learning represents a paradigm shift in
machine learning, enabling collaborative model training across
distributed datasets without centralising sensitive information.
This related work overview examines the application of federated
learning in healthcare, where patient privacy and data protection
regulations pose significant challenges to traditional centralised
machine learning approaches. We explore the core concepts of
federated learning, its specific applications in medical domains,
including disease prediction, medical imaging, and drug discovery, and discuss the technical and regulatory challenges that
must be addressed. Our review of recent literature reveals that
while federated learning offers promising solutions for privacy preserving healthcare analytics, significant research gaps remain
in areas such as model interpretability, communication efficiency,
and handling non-IID medical data distributions.
\end{abstract}

\section{Introduction}
The healthcare industry generates vast amounts of sensitive
patient data that could revolutionise medical research and
patient care through machine learning applications. However,
strict privacy regulations such as GDPR and HIPAA, combined
with ethical considerations around patient confidentiality, create significant barriers to traditional centralised data analysis
approaches [1]. Federated learning emerges as a transformative solution, enabling multiple healthcare institutions to
collaboratively train machine learning models while keeping
patient data locally secured, thus addressing both regulatory
compliance and privacy concerns that have historically limited
multi-institutional medical research collaborations.
The remainder of this paper is organised as follows. Section
II examines the core concepts underlying federated learning,
including its technical architecture and privacy-preserving
mechanisms. Section III explores specific healthcare applications where federated learning has demonstrated significant
impact, from medical imaging to drug discovery. Section IV
discusses the current challenges and future research directions,
highlighting technical limitations and regulatory considerations. Finally, Section V concludes with a summary of federated learning’s transformative potential in healthcare, while
acknowledging the work still needed to achieve widespread
adoption. 

\section{Core Concepts}
Federated learning fundamentally differs from traditional
machine learning by bringing the computation to the data
rather than centralising data for analysis [2]. In this distributed
learning paradigm, a global model is trained collaboratively
by aggregating locally computed updates from participating
institutions, each maintaining complete control over their
patient data. The process typically involves a central server
coordinating training by sending model parameters to participating hospitals or clinics, which then train the model on their
local data and return only the updated parameters, not the raw
data [3].
The mathematical foundation of federated learning relies
on techniques such as Federated Averaging (FedAvg), which
aggregates model updates from multiple clients using weighted
averaging based on the size of their local datasets [4]. This approach ensures that institutions with larger patient populations
contribute proportionally to the global model while maintaining data privacy. Recent advances have introduced differential
privacy mechanisms and secure multi-party computation protocols to further enhance privacy guarantees, making federated
learning particularly suitable for sensitive medical applications
where even model updates might reveal patient information
[5].
Healthcare-specific implementations of federated learning
must address unique challenges, including heterogeneous data
distributions across institutions, varying data quality standards,
and the need for model interpretability in clinical decisionmaking [6]. Medical data often exhibits significant non-IID
(non-independent and identically distributed) characteristics
due to demographic differences, varying diagnostic equipment,
and institutional protocols, requiring specialised federated
learning algorithms that can handle such heterogeneity while
maintaining model performance

\section{Healthcare Application}
Federated learning has demonstrated remarkable success
across diverse medical applications, particularly in medical
imaging analysis. Brisimi et al. [7] pioneered the application
of federated learning for predicting hospitalisations in heart
disease patients, demonstrating that federated models could
achieve performance comparable to centralised approaches
while preserving patient privacy. Their work across multiple
Boston-area hospitals showed that federated learning could
reduce hospitalisation rates by 34\% through early intervention
recommendations based on distributed patient data analysis.
In the domain of medical imaging, federated learning has
enabled unprecedented multi-institutional collaborations for
brain tumour segmentation, COVID-19 detection, and cancer
diagnosis [4], [5]. The ability to train deep learning models on
diverse imaging datasets from multiple hospitals without data
sharing has proven particularly valuable during the COVID-19
pandemic, where rapid model development across international
boundaries was crucial. Studies have shown that federated
learning models trained on chest X-rays from hospitals across
different continents achieved 94\% accuracy in COVID-19
detection, surpassing models trained on single-institution data
[6].
Drug discovery and pharmacovigilance represent emerging
applications where federated learning facilitates collaboration
between pharmaceutical companies, research institutions, and
regulatory bodies [1], [3]. By enabling secure analysis of
adverse drug reactions across multiple databases without exposing proprietary information or patient records, federated
learning accelerates the identification of drug safety signals
while maintaining competitive advantages and regulatory compliance. Recent implementations have demonstrated the ability
to predict drug-drug interactions with 89\% accuracy using federated learning across pharmaceutical databases, significantly
improving upon traditional pharmacovigilance methods.

\section{Challenges \& Future Directions}
Despite promising advances, federated learning in healthcare faces substantial technical and regulatory challenges
that require continued research attention. Communication efficiency remains a critical bottleneck, as medical models often
require frequent parameter updates between institutions with
varying network capabilities, and the large size of medical
imaging models can make federated training prohibitively
expensive in terms of bandwidth [2].
The interpretability of federated learning models poses
unique challenges in clinical settings where healthcare
providers require explanations for model predictions to ensure
patient safety and maintain trust [3]. Future research must
focus on developing explainable federated learning techniques
that can provide institution-specific interpretations while maintaining the privacy guarantees that make federated learning
attractive for healthcare applications. Furthermore, addressing
the statistical heterogeneity inherent in medical data across
different populations and healthcare systems remains an open
research question that will determine the practical applicability
of federated learning in global health initiatives.

\section{Conclusion}
This related work overview has examined the transformative
potential of federated learning in healthcare, demonstrating
how this technology addresses critical privacy and regulatory
challenges while enabling collaborative medical research. The
successful applications in disease prediction, medical imaging,
and drug discovery validate federated learning as a viable
approach for privacy-preserving healthcare analytics. However,
significant challenges in communication efficiency, model interpretability, and handling heterogeneous medical data require
continued research efforts. As healthcare systems increasingly
recognise the value of collaborative learning while maintaining
data sovereignty, federated learning will likely become a
fundamental technology for advancing medical research and
improving patient care globally. 

\section{Appendix}



\end{document}
