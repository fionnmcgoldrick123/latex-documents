\documentclass[a4paper,twocolumn]{article}
\usepackage{graphicx} % Required for inserting images
\usepackage{hyperref}
\usepackage[style=ieee]{biblatex}

\title{Research Methods - Tutorial 2}
\author{Fionn McGoldrick}
\date{September 2025}

\addbibresource{refs.bib}

\begin{document}

\maketitle

\section{Introduction}
This document is a sample solution for Tutorial Exercise 2. Its brief is available as a PDF on our Moodle page. 

\section{ Literature Research}
When writing your literature review and final
project dissertation, base your work mostly on recent peer-reviewed references, as these are reliable
sources.
For example, when writing a literature review
about Evolutionary Algorithms (EA) applied to
Machine Learning (ML), reading the Wikipedia
page about EA \cite{wikipedia_evolutionary_algorithm_2025} or a blog \cite{masood_quest_algorithm_iii_2025} could help you
start understanding the basics and give you some
pointers to peer-reviewed references. However, you
should not cite such web pages because they lack
editorial oversight and quality control, as nonexperts may contribute content and cannot be verified for accuracy. In other words, these are not
peer-reviewed sources.
Textbooks \cite{de2006evolutionary} can provide a good foundation in
knowledge and context, but be cautious about citing older editions, as the field evolves rapidly. Additionally, not all textbooks undergo the same rigorous peer-review process as journal articles. Always prioritise citing recent peer-reviewed papers,
particularly primary sources that report original research.
Secondary sources, such as survey \cite{banzhaf2023handbook} or review
papers, are valuable for understanding the broader
context, as they synthesise existing research to help
identify key themes and primary sources. However, secondary research citations should complement, not replace, primary ones. For example,
while \textcite{banzhaf2023handbook} investigate the role of EA in solving different ML
challenges, Zhang, Sun, Wang, et al. \textcite{11043120} propose
a new generator of optimisation benchmark problems for EA.
ArXiv preprints can provide access to recent developments \cite{novikov2025alphaevolvecodingagentscientific}, but should be used cautiously and
kept to a minimum, as they’re not peer-reviewed.
Always prioritise peer-reviewed sources as the foundation of your literature review. ArXiv papers
may be appropriate only when they represent genuinely influential work with no peer-reviewed equivalent, but ensure they don’t dominate your references.G


\printbibliography
\end{document}
