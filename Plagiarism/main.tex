\documentclass[a4paper,twocolumn]{article}
\usepackage{graphicx}
\usepackage{hyperref}
\usepackage{biblatex}
\usepackage{cleveref}
\usepackage[top=0.2in]{geometry}

\title{Research Methods - Tutorial 3}
\author{Fionn McGoldrick}
\date{October 2025}

\addbibresource{refs.bib}

\begin{document}

\maketitle

\section{Introduction}
\label{sec:intro}

This document is a sample solution for Tutorial Ex
ercise 3. Its brief is available as a PDF on Moodle.
 Section \cref{sec:direct-copying} demonstrates direct copying and its
 high similarity scores, Section \cref{sec:poor-paraphrasing} shows poor para
phrasing techniques that still result in problematic
 similarity levels, Section \cref{sec:proper-paraphrasing} illustrates effective para
phrasing methods, and Section \cref{sec:direct-quotation} covers proper use
 of direct quotations. Finally, Section \cref{sec:conclusion} concludes
 with key learning points about academic integrity
 and plagiarism detection.
 
\section{Direct Copying}
\label{sec:direct-copying}

 The following text was copied exactly from \cite{LeCun2015}:

 \begin{quote}
“Deep learning allows computational1
models that are composed of multiple
 processing layers to learn representations
 of data with multiple levels of abstrac
tion. These methods have dramatically
 improved the state-of-the-art in speech
 recognition, visual object recognition, ob
ject detection and many other domains
 such as drug discovery and genomics.
 Deep learning discovers intricate structure
 in large data sets by using the backprop
agation algorithm to indicate how a ma
chine should change its internal parame
ters that are used to compute the repre
sentation in each layer from the represen
tation in the previous layer. Deep convo
lutional nets have brought about break
throughs in processing images, video,
 speech and audio, whereas recurrent nets
 have shone light on sequential data such
 as text and speech.”
\end{quote}

 As shown in Figure \ref{fig:ss1}, the similarity score is
 100\%, indicating that virtually all of the text
 matches existing sources. This demonstrates that
 direct copying without quotation marks or proper
 attribution constitutes plagiarism. Turnitin high
lighted the entire passage in red, showing that it
 matches the original source exactly. This level of
 similarity would be unacceptable in academic work.
 
\section{Poor Paraphrasing}
\label{sec:poor-paraphrasing}

The following represents an attempt at paraphras
ing with minimal changes:

\begin{quote}
\textit{Deep learning enables computational models
 composed of multiple processing layers to acquire
 data representations with multiple abstraction lev
els. These approaches have significantly enhanced
 the state-of-the-art in speech recognition, visual
 object recognition, object detection and numerous
 other domains including drug discovery and ge
nomics. Deep learning identifies complex structure
 in large datasets by utilizing the backpropagation
 algorithm to show how a machine should modify
 its internal parameters used to calculate the repre
sentation in each layer from the representation in
 the prior layer. Deep convolutional networks have
 achieved breakthroughs in processing images, video,
 speech and audio, while recurrent networks have il
luminated sequential data such as text and speech.}
\end{quote}

Figure \ref{fig:ss2} shows a similarity score of 80%, which
 is still very high despite the synonym substitutions.
 Turnitin identified most of the sentence structures
 and key phrases as matching the original source.
 Simply replacing words with synonyms while main
taining the same sentence structure and order of
 ideas is insufficient to avoid plagiarism. The text
 still follows the original too closely and lacks origi
nal thought or restructuring


\begin{figure*}
    \centering
    \includegraphics[width=\textwidth]{ss1.png}
    \caption{Turnitin similarity report showing 100\% similarity for direct copying}
    \label{fig:ss1}
\end{figure*}

\begin{figure*}
    \centering
    \includegraphics[width=\textwidth]{ss2.png}
    \caption{ Turnitin similarity report showing 80\% similarity for poor paraphrasing.}
    \label{fig:ss2}
\end{figure*}



\section{Proper Paraphrasing}
\label{sec:proper-paraphrasing}

 The following demonstrates effective paraphrasing
 with original structure and expression:

\begin{quote}
\textit{Modern artificial intelligence has been trans
formed by deep learning technologies that em
ploy hierarchical computational architectures to ex
tract meaningful patterns from complex datasets.
 These advanced systems have revolutionized perfor
mance across diverse applications, from medical re
search to language processing and computer vision.
 The fundamental approach involves training neural
 networks through iterative parameter adjustment,
 where each processing level builds upon previous
 representations to create increasingly sophisticated
 data interpretations. While convolutional architec
tures excel at spatial data analysis such as image
 and video processing, recurrent models demonstrate
 superior performance with temporal sequences in
cluding natural language and audio signals.} 
\end{quote}

 As demonstrated in Figure \ref{fig:ss3}, the similarity score
 dropped to 0\%, indicating successful paraphras
ing. The paraphrased version maintains the origi
nal meaning while using different sentence struc
tures, vocabulary, and organisation. Only some
 technical terms might be highlighted in other cases,
 which is expected since domain-specific terminol
ogy is often unavoidable. This approach shows a
 genuine understanding of the source material while
 presenting it in an original way. Note that similar
ity scores for effective paraphrasing can vary sig
nificantly depending on factors such as technical
 terminology usage and writing style

\section{Direct Quotation}
\label{sec:direct-quotation}

 The following paragraph incorporates a direct quo
tation with proper formatting:

Research shows that “deep learning allows com
putational models that are composed of multiple
 processing layers to learn representations of data
 with multiple levels of abstraction” \cite{LeCun2015}, fundamen
tally changing how we approach artificial intelli
gence. This transformation has been particularly
 evident in the development of advanced neural net
works, which utilize hierarchical processing to ex
tract meaningful patterns from complex datasets.
 The impact extends across multiple domains, from
 computer vision systems to natural language pro
cessing applications.

 Figure \ref{fig:ss4} shows a similarity score of 32\%, demon
strating that proper quotation usage significantly
 reduces similarity concerns. The quoted portion is
 clearly identified with quotation marks and attri
bution, while the surrounding text is original. Tur
nitin recognises this as appropriate academic prac
tice, highlighting only the quoted material. This
 approach allows authors to incorporate key phrases
 from sources while maintaining academic integrity
 through proper attribution and substantial original
 content. However, quotations should be used spar
ingly and only when the original wording is par
ticularly important or cannot be effectively para
phrased. Excessive quotation usage can make your
 writing appear less original and may suggest 
insufficient understanding of the source material

\section{Conclusion}
\label{sec:conclusion}

 This exercise discusses the importance of under
standing plagiarism detection tools and proper aca
demic writing practices. Direct copying results
 in unacceptably high similarity scores, while poor
 paraphrasing that only substitutes synonyms re
mains problematic. Effective paraphrasing requires
 substantial restructuring and original expression
 of ideas. When direct quotation is appropriate,
 proper formatting and attribution ensure academic
 integrity while allowing the use of key phrases from
 sources.

\printbibliography

\begin{figure*}
    \centering
    \includegraphics[width=\textwidth]{ss3.png}
    \caption{Turnitin similarity report showing 0\% similarity for proper paraphrasing}
    \label{fig:ss3}
\end{figure*}

\begin{figure*}
    \centering
    \includegraphics[width=\textwidth]{ss4.png}
    \caption{Turnitin similarity report showing 32\% similarity for proper quotation usage.}
    \label{fig:ss4}
\end{figure*}

\end{document}
